\documentclass[
  bibliography=totoc,     % Literatur im Inhaltsverzeichnis
  captions=tableheading,  % Tabellenüberschriften
  titlepage=firstiscover, % Titelseite ist Deckblatt
]{scrartcl}

% Paket float verbessern
\usepackage{scrhack}

% Warnung, falls nochmal kompiliert werden muss
\usepackage[aux]{rerunfilecheck}

% deutsche Spracheinstellungen
\usepackage{polyglossia}
\setmainlanguage{german}

% unverzichtbare Mathe-Befehle
\usepackage{amsmath}
% viele Mathe-Symbole
\usepackage{amssymb}
% Erweiterungen für amsmath
\usepackage{mathtools}

% Fonteinstellungen
\usepackage{fontspec}
% Latin Modern Fonts werden automatisch geladen

\usepackage[
  math-style=ISO,    % ┐
  bold-style=ISO,    % │
  sans-style=italic, % │ ISO-Standard folgen
  nabla=upright,     % │
  partial=upright,   % ┘
  warnings-off={           % ┐
    mathtools-colon,       % │ unnötige Warnungen ausschalten
    mathtools-overbracket, % │
  },                       % ┘
]{unicode-math}

% traditionelle Fonts für Mathematik
\setmathfont{Latin Modern Math}
\setmathfont{XITS Math}[range={scr, bfscr}]
\setmathfont{XITS Math}[range={cal, bfcal}, StylisticSet=1]

% Zahlen und Einheiten
\usepackage[
  locale=DE,                 % deutsche Einstellungen
  separate-uncertainty=true, % immer Fehler mit \pm
  per-mode=reciprocal,       % ^-1 für inverse Einheiten
]{siunitx}

% chemische Formeln
\usepackage[
  version=4,
  math-greek=default, % ┐ mit unicode-math zusammenarbeiten
  text-greek=default, % ┘
]{mhchem}

% richtige Anführungszeichen
\usepackage[autostyle]{csquotes}

% schöne Brüche im Text
\usepackage{xfrac}

% Standardplatzierung für Floats einstellen
\usepackage{float}
\floatplacement{figure}{htbp}
\floatplacement{table}{htbp}

% Floats innerhalb einer Section halten
\usepackage[
  section, % Floats innerhalb der Section halten
  below,   % unterhalb der Section aber auf der selben Seite ist ok
]{placeins}

% Seite drehen für breite Tabellen
\usepackage{pdflscape}

% Captions schöner machen.
\usepackage[
  labelfont=bf,        % Tabelle x: Abbildung y: ist jetzt fett
  font=small,          % Schrift etwas kleiner als Dokument
  width=0.9\textwidth, % maximale Breite einer Caption schmaler
]{caption}
% subfigure, subtable, subref
\usepackage{subcaption}

% Grafiken können eingebunden werden
\usepackage{graphicx}
% größere Variation von Dateinamen möglich
\usepackage{grffile}

% schöne Tabellen
\usepackage{booktabs}

% Verbesserungen am Schriftbild
\usepackage{microtype}

% Literaturverzeichnis
\usepackage[
  backend=biber,
]{biblatex}
% Quellendatenbank
\addbibresource{lit.bib}
\addbibresource{programme.bib}

% Hyperlinks im Dokument
\usepackage[
  unicode,        % Unicode in PDF-Attributen erlauben
  pdfusetitle,    % Titel, Autoren und Datum als PDF-Attribute
  pdfcreator={},  % ┐ PDF-Attribute säubern
  pdfproducer={}, % ┘
]{hyperref}
% erweiterte Bookmarks im PDF
\usepackage{bookmark}

% Trennung von Wörtern mit Strichen
\usepackage[shortcuts]{extdash}

% für bra und ket
\usepackage{braket}

\author{
  Ksenia Klassen%
  \texorpdfstring{
    \\
    \href{mailto:ksenia.klasser@udo.edu}{ksenia.klassen@udo.edu}
  }{}%
  \texorpdfstring{\and}{, }
  Dag-Björn Hering%
  \texorpdfstring{
    \\
    \href{mailto:dag.hering@udo.edu}{dag.hering@udo.edu}
  }{}%
  \texorpdfstring{\and}{, }
  Henning Ptaszyk%
  \texorpdfstring{
    \\
    \href{mailto:henning.ptaszyk@udo.edu}{henning.ptaszyk@udo.edu}
  }{}%
}



\title{0. Übungsblatt}

\begin{document}

\maketitle
\thispagestyle{empty}
\newpage

%\printbibliography


\section{Aufgabe 1}
\subsection{a)}
\label{sec:1a}


\begin{figure}
  \centering
  \includegraphics[width=0.7\textwidth]{plot 1a).pdf}
  \caption{Plot für die Funktion aus a).}
  \label{fig:a}
\end{figure}

\subsection{b)}
\label{sec:1b}

\begin{figure}
  \centering
  \includegraphics[width=0.7\textwidth]{plot 1b).pdf}
  \caption{Plot für die Funktion aus b).}
  \label{fig:b}
\end{figure}
\FloatBarrier

\subsection{c)}
\label{sec:1c}

\begin{figure}
  \centering
  \includegraphics[width=0.7\textwidth]{plot 1c).pdf}
  \caption{Plot für die Funktion aus c).}
  \label{fig:c}
\end{figure}

\subsection{Interpretation}
Der Plott \ref{fig:a} liefert als einziges den richtigen Verlauf der Funktion
Wohingegen \ref{fig:b} und \ref{fig:c}  keinen klaren Verlauf erkennen lassen.
Dies liegt an den Rundungsfehlern, die das Programm
bei den Auswertung der Formeln aus \ref{sec:b} und \ref{sec:c} macht.

\section{Aufgabe 2}
\subsection{a)}

\begin{align}
  \lim_{x \to 0}\left(\frac{\sqrt{9-x}-3}{x}\right)\stackrel{\mathrm{l'Hopital}}{=}
  \lim_{x \to 0}\left(\frac{-\frac{1}{2\sqrt{9-x}}}{1}\right)=-\frac{1}{6}
\end{align}


\subsection{b)}

\begin{figure}
  \centering
  \includegraphics[width=0.7\textwidth]{plot 2b).pdf}
  \caption{Plot für den Grenzwert.}
  \label{fig:2b}
\end{figure}

Die berechneten Werte springen nach $x=10^{-15}$ auf $0$ folglich liefet die
numerische Auswertung das falsche Ergebnis. Dies liegt daran, dass
der Term $9-x$ bei Zahlen die kleiner sind als $10^{-15}$ auf $9$ gerundet wird.
Mit diesem Ergebnis wird der Zähler $0$ und damit auch die ganze Funktion.

\section{Aufgabe 3}
Beide Funktionen sind algebraisch konstant $\frac{2}{3}$.
Eine Abweichung von mehr als $1\%$
liegt über einem beträgsmäßigen Fehler von ungefähr
\begin{align*}
  0,0067
\end{align*}
vor.

\subsection{a)/c)}
\begin{figure}
  \centering
  \includegraphics[width=0.7\textwidth]{plot 3a)log.pdf}
  \caption{Betragsmäßige Abweichung von der Funktion $f(x)$.}
  \label{fig:3a}
\end{figure}
\FloatBarrier
\begin{align*}
\intertext{\Rightarrow In einem Bereich von:}
-4\cdot10^{4}\leq x \leq 4\cdot10^{4}
\intertext{Weicht das numerische Ergebnis vom Algebraischen um nicht mehr als $1\%$ ab}
\intertext{Die Abweichung ist ungefähr zwischen}
-5\cdot10^{-3}\leq x \leq 5\cdot10^{-3}
\intertext{gleich 0.}
\end{align*}

\subsection{b)/c)}

\begin{figure}
  \centering
  \includegraphics[width=0.7\textwidth]{plot 3b)log.pdf}
  \caption{Betragsmäßige Abweichung von der Funktion $f(x)$.}
  \label{fig:3a}
\end{figure}
\FloatBarrier
\begin{align*}
\intertext{\Rightarrow In einem Bereich von:}
-5\cdot10^{-5}\leq x \leq 5\cdot10^{-5}
\intertext{Weicht das numerische Ergebnis vom Algebraischen um nicht mehr als $1\%$ ab}
\intertext{Die Abweichung ist ungefähr bei}
\left|x\right|\geq 2\cdot10^{-4}
\intertext{gleich 0.}
\end{align*}

\newpage
\section{Aufgabe4}
\label{sec:a4}

\subsection{a)}
\label{subsec:a4a}
Nein, die angegebene Gleichung ist nicht stabil, denn für Werte von $\Theta$ in der nahen Umgebung ganzzahliger Vielfacher von $\pi$ werden im Nenner zwei
nahezu gleichgroße Zahlen voneinander abgezogen. Dies ist der Fall, da $\beta^{2}$ für
\\
\begin{equation}
  \label{eqn:a4af1}
  E_\text{e} = \SI{50}{\giga\electronvolt}
\end{equation}
einen Wert sehr nahe $1$ annimmt, sodass immmer wenn gilt
\\
\begin{equation}
  \label{eqn:a4af2}
  \lvert \cos\left( \theta \right) \rvert = 1,
\end{equation}
das besagte Stabilitätsproblem auftritt.

\subsection{b)}
\label{subsec:a4b}
Das in Abschnitt \ref{subsec:a4a} beschriebene Stabilitätsproblem lässt sich durch die Folgende algebraische Umformung des Nenners weitgehend beheben:
\\
\begin{align}
  \label{eqn:a4bf1}
  &1 - \beta^{2} \cdot \cos\left( \theta \right)^{2}\\
  &=\sin\left( \theta \right)^{2} + \cos\left( \theta \right)^{2} - \beta^{2} \cdot \cos\left( \theta \right)^{2} \\
  &=\sin\left( \theta \right)^{2} + (1 - \beta^{2}) \cdot \cos\left( \theta \right)^{2}\\
  &=\sin\left( \theta \right)^{2} + \gamma^{-2} \cdot \cos\left( \theta \right)^{2}
\end{align}

Somit ergibt sich dann für folgender Ausdruck für den Wirkungsquerschnitt:
\\
\begin{equation}
  \label{eqn:a4bf2}
  \frac{\text{d}\sigma}{\text{d}\Omega} = \frac{\alpha^{2}}{s} \cdot \frac{2 + \sin\left( \theta \right)}{\sin\left( \theta \right)^{2} + \gamma^{-2} \cdot \cos\left( \theta \right)^{2}}.
\end{equation}

In der in Gleichung \eqref{eqn:a4bf2} gezeigten Darstellung der Formel, wird im Nenner nun eine Addition ausgeführt, was die Stabilitätsprobleme verringert.

\subsection{c)}
\label{subsec:a4c}
Um die in Abschnitt \ref{subsec:a4b} beschriebene Stabilitätsverbesserung zu verifizieren, werden nun beide Funktionen in einem sehr kleinen Bereich
um $pi$ herum geplottet. In Abbildung \ref{fig:a4A1} sind die zugehörigen Graphen gezeigt. Anhand dieser ist klar zu erkennen, dass die Rundungsfehler,
welche sich in Form der Treppen äußern,
eindeutig unterdrückt wurden.
\\

\FloatBarrier
\begin{figure}
  \centering
  \includegraphics[width=\textwidth]{plot 4c).pdf}
  \caption{Graphische Darstellung der ursprünglichen Funktion sowie der Stabilisierten .}
  \label{fig:a4A1}
\end{figure}
\FloatBarrier

\subsection{d)}
Die Konditionszahl $K$ wird über den folgenden Zusammenhang berechnet:
\\
\begin{equation}
  \label{eqn:a4df1}
  K = \lvert \theta \cdot \frac{\text{d}f(\theta)}{\text{d}\theta} \cdot \frac{1}{f(\theta)}\rvert.
\end{equation}
Für die Ableitung des differentiellen Wirkungsquerschnitts nach $\theta$ gilt
\\
\begin{equation}
  \label{eqn:a4df2}
   \frac{\text{d}f(\theta)}{\text{d}\theta} = \frac{\left(1-3\beta\right) \cdot \sin\left( 2\theta \right)}{\left( 1 - \beta^{2} \cdot \cos\left( \theta \right)^{2} \right)^{2}}
\end{equation}

Damit ergibt sich dann für die Konditionszahl
\\
\begin{equation}
  \label{eqn:a4df3}
  \lvert \frac{\left( 1 - 3\beta^{2} \right) \cdot \left( 1 - \beta^{2} \cdot \cos\left( \theta \right)^{2} \right)^{2} \cdot \sin\left( 2\theta \right)}{\left( \sin\left( \theta \right)^{2} + 2\right) \cdot \left( \beta^{2} \cdot \cos\left( \theta \right)^{2} - 1 \right)^{2}} \rvert.
\end{equation}

\subsection{e)}
Um zu veranschaulichen für welche Werte von $\theta$ eine gute oder schlechte Konditionierung vorliegt, ist die Konditionszahl $K\left( \theta \right)$
in Abbildung \ref{fig:a4eA1} für $\theta$ zwischen $0$ und $\pi$ gegen $\theta$ aufgetragen.
\\
Es ist zu erkennen, dass das Problem gut Konditioniert ist bis auf Werte sehr nahe $\pi$, wo es eine sehr schlechte Konditionierung aufweist.
\FloatBarrier
\begin{figure}
  \centering
  \includegraphics[width=\textwidth]{plot 4e).pdf}
  \caption{Konditionszahl in Abhängikeit von Theta.}
  \label{fig:a4eA1}
\end{figure}
\FloatBarrier



\end{document}
