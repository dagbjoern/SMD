\input{header.tex}


\title{0. Übungsblatt}

\begin{document}

\maketitle
\thispagestyle{empty}
\newpage

%\printbibliography


\section{Aufgabe 1}
\subsection{a)}
\label{sec:1a}


\begin{figure}
  \centering
  \includegraphics[width=0.7\textwidth]{plot 1a).pdf}
  \caption{Plot für die Funktion aus a).}
  \label{fig:a}
\end{figure}

\subsection{b)}
\label{sec:1b}

\begin{figure}
  \centering
  \includegraphics[width=0.7\textwidth]{plot 1b).pdf}
  \caption{Plot für die Funktion aus b).}
  \label{fig:b}
\end{figure}
\FloatBarrier

\subsection{c)}
\label{sec:1c}

\begin{figure}
  \centering
  \includegraphics[width=0.7\textwidth]{plot 1c).pdf}
  \caption{Plot für die Funktion aus c).}
  \label{fig:c}
\end{figure}

\subsection{Interpretation}
Der Plott \ref{fig:a} liefert als einziges den richtigen Verlauf der Funktion
Wohingegen \ref{fig:b} und \ref{fig:c}  keinen klaren Verlauf erkennen lassen.
Dies liegt an den Rundungsfehlern, die das Programm
bei den Auswertung der Formeln aus \ref{sec:b} und \ref{sec:c} macht.

\section{Aufgabe 2}
\subsection{a)}

\begin{align}
  \lim_{x \to 0}\left(\frac{\sqrt{9-x}-3}{x}\right)\stackrel{\mathrm{l'Hopital}}{=}
  \lim_{x \to 0}\left(\frac{-\frac{1}{2\sqrt{9-x}}}{1}\right)=-\frac{1}{6}
\end{align}


\subsection{b)}

\begin{figure}
  \centering
  \includegraphics[width=0.7\textwidth]{plot 2b).pdf}
  \caption{Plot für den Grenzwert.}
  \label{fig:2b}
\end{figure}

Die berechneten Werte springen nach $x=10^{-15}$ auf $0$ folglich liefet die
numerische Auswertung das falsche Ergebnis. Dies liegt daran, dass
der Term $9-x$ bei Zahlen die kleiner sind als $10^{-15}$ auf $9$ gerundet wird.
Mit diesem Ergebnis wird der Zähler $0$ und damit auch die ganze Funktion.

\section{Aufgabe 3}
Beide Funktionen sind algebraisch konstant $\frac{2}{3}$.
Eine Abweichung von mehr als $1\%$
liegt über einem beträgsmäßigen Fehler von ungefähr
\begin{align*}
  0,0067
\end{align*}
vor.

\subsection{a)/c)}
\begin{figure}
  \centering
  \includegraphics[width=0.7\textwidth]{plot 3a)log.pdf}
  \caption{Betragsmäßige Abweichung von der Funktion $f(x)$.}
  \label{fig:3a}
\end{figure}
\FloatBarrier
\begin{align*}
\intertext{\Rightarrow In einem Bereich von:}
-4\cdot10^{4}\leq x \leq 4\cdot10^{4}
\intertext{Weicht das numerische Ergebnis vom Algebraischen um nicht mehr als $1\%$ ab}
\intertext{Die Abweichung ist ungefähr zwischen}
-5\cdot10^{-3}\leq x \leq 5\cdot10^{-3}
\intertext{gleich 0.}
\end{align*}

\subsection{b)/c)}

\begin{figure}
  \centering
  \includegraphics[width=0.7\textwidth]{plot 3b)log.pdf}
  \caption{Betragsmäßige Abweichung von der Funktion $f(x)$.}
  \label{fig:3a}
\end{figure}
\FloatBarrier
\begin{align*}
\intertext{\Rightarrow In einem Bereich von:}
-5\cdot10^{-5}\leq x \leq 5\cdot10^{-5}
\intertext{Weicht das numerische Ergebnis vom Algebraischen um nicht mehr als $1\%$ ab}
\intertext{Die Abweichung ist ungefähr bei}
\left|x\right|\geq 2\cdot10^{-4}
\intertext{gleich 0.}
\end{align*}

\newpage
\section{Aufgabe4}
\label{sec:a4}

\subsection{a)}
\label{subsec:a4a}
1.) Tokenisierung: Segmentierung eines Textes in Einheiten, z.b. einzelne Wörter oder Satzteile.
    Das Ziel ist die Entfernung von unwichtigen Tokens, z.B. Füllwörtern oder Satzzeichen.\\
2.) Fehlerhafte Daten entfernen: Wenn sich Ausreißer unter den Daten befinden, z.B. Gewicht von 400kg bei Menschen.\\
3.) Default Werte verenden: es werden Default Werte anstelle der Fehlerhaften Daten verwendet.\\
4.) Ableiten aus Anderen Daten: aus Daten können korrekte Werte abgeleitet werden (Anrede aus Vornamen)\\
\subsection{b)}
\label{subsec:a4b}
 Es ist günstig Attribute auf einen einheitlichen Wertebereich zu normieren, für eine leichtere Weiterverarbeitung.
 (Z.B: Firmenzusatz: e.Kfr, e.Kfm zusammengefasst zu e.K)

\subsection{c)}
\label{subsec:a4c}
Lücken in den Datensätzen müssen sinnvoll ersetzt oder evtl. gelöscht werden.

\subsection{d)}
\label{subsec:a4d}
Beim Zusammenführen von Datensätzen muss beachtet werden, dass die Teile zueinander passen und kombiniert werden können.



\end{document}
