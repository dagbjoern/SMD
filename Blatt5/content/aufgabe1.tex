\newpage
\section{Aufgabe 1}
\label{sec:a1}

\subsection{a)}
\label{subsec:a1a}
Zunächst wird für beide Populationen der Mittelwert berechnet.
\begin{align}
\mu_{P0}\approx
\begin{pmatrix*}[r]
   0,022\\
   3,026
\end{pmatrix*}\\
\mu_{P1}\approx
\begin{pmatrix*}[r]
   5,986\\
   3,026
\end{pmatrix*}
\end{align}

\subsection{b)}
\label{subsec:a1b}
Die Kovantianzmatrix $S_{P0}$ und $S_{P1}$ berechnen sich über
\begin{align}
S_j=\sum_i^{n_j}(\vec x- \vec \mu_j)(\vec x- \vec \mu_j)^T
\intertext{zu}
S_{P0}\approx
\begin{pmatrix*}[r]
     122904 &  81979\\
     81979  &  67449
\end{pmatrix*}
\intertext{und}
S_{P1}\approx
\begin{pmatrix*}[r]
    122344 &  73118\\
     73118 &   53845
\end{pmatrix*}
\intertext{Aus $S_{P0P1}=S_{P0}+S_{P1}$ folgt}
S_{P0P1}\approx
\begin{pmatrix*}[r]
 245248  & 155097\\
 155097  & 121294
\end{pmatrix*}
\intertext{Desweiteren muss noch $S_B$ über}
S_B=(\vec \mu_1-\vec \mu_2)(\vec \mu_1-\vec \mu_2)^T
\intertext{berechnet werden.}
\Rightarrow S_B\approx
\begin{pmatrix*}[r]
 35.562 &   0.432\\
 0.432  & 0.005
\end{pmatrix*}
\end{align}

\subsection{c)}
\label{subsec:a1c}

\subsection{d)}
\label{subsec:a1d}

\subsection{e)}
\label{subsec:a1e}

\subsection{f)}
\label{subsec:a1f}

\subsection{g)}
\label{subsec:a1g}

\subsection{h)}
\label{subsec:a1h}
