\newpage
\section{Aufgabe 2}

\subsection{a)}
\label{sec:A2a}
Die Hypotese A sagt einen Wert von $31,3\si{\milli\electronvolt}$
vorraus.
Es wird ein $\chi^2$-Test mit Hilfe der Formel \ref{eqn:chi}
duchgeführt.
\begin{align}
  \chi^2=\sum_{i=1}^n\frac{(y_i-f(x_i))^2}{(\sigma_i^{Modell})^2}\label{eqn:chi}
\end{align}
Werden nun die Messwerte und die Hypothese und $\sigma=0,5\si{\milli\electronvolt}$ in \eqref{eqn:chi} eingesetzt ergibt sich:
\begin{align}
  \chi&=6,08 \\
  \Rightarrow \alpha&\approx50\%
\end{align}
Die These wird bei
einer Größeren Signifikans als $5\%$ verworfen.
Da \alpha bei der Hypotese A $50\%$ ist folglich größer als $5\%$ wird A verworfen.
\subsection{b)}
Das vorgehen ist identisch zu \ref{sec:A2a} nur
das diesmal eine Hypothese B mit einem Wert von $30,7$
benutzt wird.
Für $\chi$ ergibt sich aus \eqref{eqn:chi}:
\begin{align}
  \chi&=21,92 \\
  \Rightarrow \alpha&\approx 1\%
\end{align}
Da $\alpha$ kleiner als $5\%$ ist wird die Hypothese B nicht verworfen.
