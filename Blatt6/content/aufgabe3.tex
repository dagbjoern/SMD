\newpage
% \section{Aufgabe3}
% \label{sec:a3}
%


\begin{figure}
  \centering
  \includegraphics[width=\textwidth]{PhotoScan.jpg}
  \caption{}.
  \label{fig:1}
\end{figure}


\FloatBarrier

\paragraph{c)}
In den Abbildungen \ref{fig:Temperatur}- \ref{fig:Luftfeuchigkeit}
ist der Informationsgewinn
in Abbhängigkeit der jeweiligen Schnitte auf den
unterschiedlichen Attributen aufgetragen.


\begin{figure}
  \centering
  \includegraphics[width=0.7\textwidth]{Temperatur.pdf}
  \caption{ Der Informationsgewinn gain(a) in Abhängigkeit von dem Schnitt $s$ auf dem Attribut a=Temperatur .}
  \label{fig:Temperatur}
\end{figure}


\begin{figure}
  \centering
  \includegraphics[width=0.7\textwidth]{Vorhersage.pdf}
  \caption{ Der Informationsgewinn gain(a) in Abhängigkeit von dem Schnitt $s$ auf dem Attribut a=Wettervorhersage .}
  \label{fig:Vorhersage}
\end{figure}


\begin{figure}
  \centering
  \includegraphics[width=0.7\textwidth]{Luftfeuchigkeit.pdf}
  \caption{ Der Informationsgewinn gain(a) in Abhängigkeit von dem Schnitt $s$ auf dem Attribut a=Luftfeuchigkeit .}
  \label{fig:Luftfeuchigkeit}
\end{figure}
\FloatBarrier

\paragraph{d)}

Ein Schnitt s auf dem Attribut Temperatur liefert für
\begin{align}
 s=28,7
  \intertext{den größten Informationsgewinn gain(Temperatur) von:}
  \text{gain(a)} \approx 0,11
\end{align}
und eignet sich somit am besten zum Trennen der Daten.
