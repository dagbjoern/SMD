\section{Aufgabe 1}
\label{sec:a1}

\subsection{a)}
\label{subsec:a1a}
Haben die Attribute stark verschiedene Größenordnungen,
ist es sehr wichtig die Attribute zu normieren. So wird vermieden
,dass Attribute, die eine vergleichsweise größere Größenordnung
haben, stärker berücksichtigt werden als andere. Dies würde
ohne Normierung passieren, da Abstände gebildet werden.

\subsection{b)}
\label{subsec:a1b}
Der \textbf{k-NN-Algorithmus} speichert beim Lernen
einfach die Abständsvektoren der Trainingsdaten ab. Da
somit also eigentlich nichts mit den Trainingsdaten passiert
kann der Algorithmus als \textbf{"lazy-learner"} bezeichnet
werden. Somit sind beschränkt sich die Laufzeit beim Lernen auf
die Zeit, die benötigt wird um die Trainingsdaten abzuspeichern.
In der Anwendungsphase müssen jeweils die Abstände der
zu klassifizierenden Daten zu den Trainingsdaten bestimmt
werden und anschließend noch sortiert werden.
Diese Eigenschaften unterscheiden sich stark von anderen
Algorithmen wie zB. einem \textbf{Random-Forest}, der
mehr Aufwand in das Lernen steckt und dafür beim klassifizieren
schneller ist.


\subsection{c)}
\label{subsec:a1c}
Siehe \textbf{aufgabe1.py}.

\subsection{d)}
Für die zu ermittelnden Größen ergeben sich:
\\
\begin{align*}
  \text{Reinheit} &= \SI{0.645}{}\\
  \text{Effizienz} &= \SI{0.942}{}\\
  \text{Signifikanz} &= \SI{0.314}{}\\
\end{align*}

\subsection{e)}
Nach Logarithmieren der Hits ergibt sich:
\\
\begin{align*}
  \text{Reinheit} &= \SI{0.645}{}\\
  \text{Effizienz} &= \SI{0.942}{}\\
  \text{Signifikanz} &= \SI{0.314}{}\\
\end{align*}
Das Ergebnis verändert sich nicht.
Obwohl man hätte erwarten können dass es sich verbessert,
weil die \textbf{Anzahl Hits} jetzt in die selbe
Größenordnung fällt wie \textbf{x} und \textbf{y}.
\subsection{f)}
\label{subsec:a1f}
Bei Verwendung von $20$ anstatt von $10$ nächsten Nachbarn, ergeben sich:
\\
\begin{align*}
  \text{Reinheit} &= \SI{0.633}{}\\
  \text{Effizienz} &= \SI{0.899}{}\\
  \text{Signifikanz} &= \SI{0.300}{}\\
\end{align*}
Es ist zu erkennen, dass sich das Ergebnis leicht verschlechtert wenn zu viele
Nachbarn hinzugezogen werden.
