\newpage
\section{Aufgabe 1}

\subsection{a)}
\label{sec:A1a}

\begin{align}
  \intertext{Nullhyphothese:} s_0=0\\
  \intertext{Maximum bestimmen von  $ \ln L $ :}
  0=\frac{ \partial }{ \partial b}(\ln L)\\
  0&=\frac{N_\text{off}}{b_0}+\frac{N_\text{on}}{b_0}+(1+\alpha)\\
  b_0&=\frac{N_\text{off}+N_\text{on}}{1+\alpha}
   \intertext{Fehler für den Parameter $b_0$:}
  G_{11}&=\frac{\partial (-\ln L)}{\partial s \partial s}=0\\
  G_{12}&=\frac{\partial (-\ln L)}{\partial s \partial b}=0\\
  G_{21}&=\frac{\partial (-\ln L)}{\partial b \partial s}=0\\
  G_{22}&=\frac{\partial (-\ln L)}{\partial b \partial b}=
 \frac{N_\text{off}+N_\text{on}}{b_0^2}\\
\Rightarrow G&=
\begin{pmatrix}
0 & 0 \\
0 & \frac{(1+\alpha)^2}{N_\text{off}+N_\text{on}}  \\
\end{pmatrix}\\
V&=G^{-1} \\
\Rightarrow \sigma_{b_0}^2&=\frac{N_\text{off}+N_\text{on}}{(1+\alpha)^2}\\
\sigma_{b_0}&=\frac{\sqrt{N_\text{off}+N_\text{on}}}{(1+\alpha)}
\end{align}

\subsection{b)}
Nun wird das Verhältnis $\lambda$ der beiden Likelihoods bestimmt.

\begin{align}
  \lambda&=\frac{L(s=0,b_0)}{L(\hat s,\hat b)}\\
  \lambda&=\symup{e}^{N_\text{off}\ln\left(\frac{b_0}{\hat b}\right)
  +N_\text{on}\ln\left(\frac{\alpha b_0}{\hat s+\alpha \hat b}\right)
  -(1+\alpha)(b_0-\hat b)+\hat s}\\
\intertext{mit}
b_0&=\frac{N_\text{off}+N_\text{on}}{1+\alpha},\quad
\hat s= N_\text{on}-\alpha N_\text{off},\quad
\hat b= N_\text{off}
\end{align}

\subsection{c)}
Für größe $N_\text{on},N_\text{off}$ ist $ D= -\ln \lambda$
$\chi^2$-verteilt mit einem Freiheitsgrad.
Folglich kann aus der ersten Zeile aus der Tabelle auf
der Seite $83$ aus dem
Vorlesungskapitel Testen die Konfidenz ablesen
werden
mit der die Nullhypothese in Einheiten von $\sigma$.
abgelehnt werden kann.
\subsection{d)}
Für die gegebenen Werte ergibt sich für
\begin{align}
D_1= 1,69
\intertext{und für }
D_2= 9,58.
\end{align}
Folglich beträgt die Signifikanz der
Nullhypothese bei der ersten Messung ungefähr $20\%$
 und bei der zweiten Messung ungefähr $1\%$
