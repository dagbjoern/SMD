\newpage
\section{Aufgabe4}
\label{sec:a4}

\subsection{a)}
\label{subsec:a4a}
Nein, die angegebene Gleichung ist nicht stabil, denn für Werte von $\Theta$ in der nahen Umgebung ganzzahliger Vielfacher von $\pi$ werden im Nenner zwei
nahezu gleichgroße Zahlen voneinander abgezogen. Dies ist der Fall, da $\beta^{2}$ für
\\
\begin{equation}
  \label{eqn:a4af1}
  E_\text{e} = \SI{50}{\giga\electronvolt}
\end{equation}
einen Wert sehr nahe $1$ annimmt, sodass immmer wenn gilt
\\
\begin{equation}
  \label{eqn:a4af2}
  \lvert \cos\left( \theta \right) \rvert = 1,
\end{equation}
das besagte Stabilitätsproblem auftritt.

\subsection{b)}
\label{subsec:a4b}
Das in Abschnitt \ref{subsec:a4a} beschriebene Stabilitätsproblem lässt sich durch die Folgende algebraische Umformung des Nenners weitgehend beheben:
\\
\begin{align}
  \label{eqn:a4bf1}
  &1 - \beta^{2} \cdot \cos\left( \theta \right)^{2}\\
  &=\sin\left( \theta \right)^{2} + \cos\left( \theta \right)^{2} - \beta^{2} \cdot \cos\left( \theta \right)^{2} \\
  &=\sin\left( \theta \right)^{2} + (1 - \beta^{2}) \cdot \cos\left( \theta \right)^{2}\\
  &=\sin\left( \theta \right)^{2} + \gamma^{-2} \cdot \cos\left( \theta \right)^{2}
\end{align}

Somit ergibt sich dann für folgender Ausdruck für den Wirkungsquerschnitt:
\\
\begin{equation}
  \label{eqn:a4bf2}
  \frac{\text{d}\sigma}{\text{d}\Omega} = \frac{\alpha^{2}}{s} \cdot \frac{2 + \sin\left( \theta \right)}{\sin\left( \theta \right)^{2} + \gamma^{-2} \cdot \cos\left( \theta \right)^{2}}.
\end{equation}

In der in Gleichung \eqref{eqn:a4bf2} gezeigten Darstellung der Formel, wird im Nenner nun eine Addition ausgeführt, was die Stabilitätsprobleme verringert.

\subsection{c)}
\label{subsec:a4c}
Um die in Abschnitt \ref{subsec:a4b} beschriebene Stabilitätsverbesserung zu verifizieren, werden nun beide Funktionen in einem sehr kleinen Bereich
um $pi$ herum geplottet. In Abbildung \ref{fig:a4A1} sind die zugehörigen Graphen gezeigt. Anhand dieser ist klar zu erkennen, dass die Rundungsfehler,
welche sich in Form der Treppen äußern,
eindeutig unterdrückt wurden.
\\

\FloatBarrier
\begin{figure}
  \centering
  \includegraphics[width=\textwidth]{plot 4c).pdf}
  \caption{Graphische Darstellung der ursprünglichen Funktion sowie der Stabilisierten .}
  \label{fig:a4A1}
\end{figure}
\FloatBarrier

\subsection{d)}
Die Konditionszahl $K$ wird über den folgenden Zusammenhang berechnet:
\\
\begin{equation}
  \label{eqn:a4df1}
  K = \lvert \theta \cdot \frac{\text{d}f(\theta)}{\text{d}\theta} \cdot \frac{1}{f(\theta)}\rvert.
\end{equation}
Für die Ableitung des differentiellen Wirkungsquerschnitts nach $\theta$ gilt
\\
\begin{equation}
  \label{eqn:a4df2}
   \frac{\text{d}f(\theta)}{\text{d}\theta} = \frac{\left(1-3\beta\right) \cdot \sin\left( 2\theta \right)}{\left( 1 - \beta^{2} \cdot \cos\left( \theta \right)^{2} \right)^{2}}
\end{equation}

Damit ergibt sich dann für die Konditionszahl
\\
\begin{equation}
  \label{eqn:a4df3}
  \lvert \frac{\left( 1 - 3\beta^{2} \right) \cdot \left( 1 - \beta^{2} \cdot \cos\left( \theta \right)^{2} \right)^{2} \cdot \sin\left( 2\theta \right)}{\left( \sin\left( \theta \right)^{2} + 2\right) \cdot \left( \beta^{2} \cdot \cos\left( \theta \right)^{2} - 1 \right)^{2}} \rvert.
\end{equation}

\subsection{e)}
Um zu veranschaulichen für welche Werte von $\theta$ eine gute oder schlechte Konditionierung vorliegt, ist die Konditionszahl $K\left( \theta \right)$
in Abbildung \ref{fig:a4eA1} für $\theta$ zwischen $0$ und $\pi$ gegen $\theta$ aufgetragen.
\\
Es ist zu erkennen, dass das Problem gut Konditioniert ist bis auf Werte sehr nahe $\pi$, wo es eine sehr schlechte Konditionierung aufweist.
\FloatBarrier
\begin{figure}
  \centering
  \includegraphics[width=\textwidth]{plot 4e).pdf}
  \caption{Konditionszahl in Abhängikeit von Theta.}
  \label{fig:a4eA1}
\end{figure}
\FloatBarrier
