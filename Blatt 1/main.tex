\input{header.tex}


\title{0. Übungsblatt}

\begin{document}

\maketitle
\thispagestyle{empty}
\newpage

%\printbibliography


\section{Aufgabe 1}
\subsection{a)}
\label{sec:1a}


\begin{figure}
  \centering
  \includegraphics[width=0.7\textwidth]{plot a).pdf}
  \caption{Plot für die Funktion aus a).}
  \label{fig:a}
\end{figure}

\subsection{b)}
\label{sec:1b}

\begin{figure}
  \centering
  \includegraphics[width=0.7\textwidth]{plot b).pdf}
  \caption{Plot für die Funktion aus b).}
  \label{fig:b}
\end{figure}
\FloatBarrier

\subsection{c)}
\label{sec:1c}

\begin{figure}
  \centering
  \includegraphics[width=0.7\textwidth]{plot c).pdf}
  \caption{Plot für die Funktion aus c).}
  \label{fig:c}
\end{figure}

\subsection{Interpretation}
Der Plott \ref{fig:a} liefert als einziges den richtigen Verlauf der Funktion
Wohingegen \ref{fig:b} und \ref{fig:c}  keinen klaren Verlauf erkennen lassen.
Dies liegt an den Rundungsfehlern, die das Programm
bei den Auswertung der Formeln aus \ref{sec:b} und \ref{sec:c} macht.

\section{Aufgabe 2}
\subsection{a)}

\begin{align}
  \lim_{x \to 0}\left(\frac{\sqrt{9-x}-3}{x}\right)\stackrel{\mathrm{l'Hopital}}{=}
  \lim_{x \to 0}\left(\frac{-\frac{1}{2\sqrt{9-x}}}{1}\right)=-\frac{1}{6}
\end{align}


\subsection{b)}

\begin{figure}
  \centering
  \includegraphics[width=0.7\textwidth]{plot 2b).pdf}
  \caption{Plot für den Grenzwert.}
  \label{fig:2b}
\end{figure}

Die berechneten Werte springen nach $x=10^{-15}$ auf $0$ folglich liefet die
numerische Auswertung das falsche Ergebnis. Dies liegt daran, dass
der Term $9-x$ bei Zahlen die kleiner sind als $10^{-15}$ auf $9$ gerundet wird.
Mit diesem Ergebnis wird der Zähler $0$ und damit auch die ganze Funktion.        

\section{Aufgabe 3}

\section{Aufgabe 4}

\end{document}
