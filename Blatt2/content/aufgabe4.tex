\newpage
\section{Aufgabe4}
\label{sec:a4}

\subsection{a)}
\label{subsec:a4a}
 Betrachtet wird die Funktion einer
 Ausgleichsgeraden
 \\
 \begin{equation}
   \label{eqn:f}
   y = a_{0} + a_{1} \cdot x.
 \end{equation}
 Dabei seien die Parameter
 \\
 \begin{align*}
   a_{0} &= \SI{1.0+-0.2}{}\\
   a_{1} &= \SI{1.0+-0.2}{}
 \end{align*}
 sowie der Korrelationskoeffizient
 \\
 \begin{align*}
   \rho = -0.8
 \end{align*}
gegeben.

Berücksichtigt man die Korrelation,
ergibt sich für die Unsicherheit von $y$
in Abhängigkeit von $x$ der Zusammenhang
\\
\begin{equation}
  \label{eqn:korr}
  \sigma_{y}(x) = \sqrt{\sigma_{a_{0}}^{2} + \left( x \cdot \sigma_{a_{1}} \right)^{2}
  + 2 \cdot \left( x \cdot \rho \cdot \sigma_{a_{0}} \cdot \sigma_{a_{1}} \right)}.
\end{equation}
Lässt man die Korrelation außer acht, so ergibt sich nach Gausßscher Fehlerfortpflanzung
\\
\begin{equation}
  \label{eqn:unkorr}
  \sigma_{y}(x) = \sqrt{\sigma_{a_{0}}^{2} + \left( x \cdot \sigma_{a_{1}} \right)^{2}}.
\end{equation}


\subsection{a)}
\label{subsec:a4a}

\subsection{b)}
\label{subsec:a4b}

\subsection{c)}
\label{subsec:a4c}
Für den Fall ohne Korrelation ergibt sich
mit Gleichungen \ref{eqn:unkorr} und \ref{eqn:f}
\\
\begin{align*}
  y(-3) &= \SI{-2.0}{}\\
  \sigma_{y}(-3) &= \pm 0.6\\
  y(0) &= \SI{1.0}{}  \\
  \sigma_{y}(0) &= \pm0.2\\
  y(3) &= \SI{4.0}{} \\
  \sigma_{y}(3) &= \pm0.6
\end{align*}
Für den Fall mit Korrelation ergibt sich mit
Gleichung \ref{eqn:korr} \ref{eqn:f}
\\
\begin{align*}
  y(-3) &= \SI{-2.0}{}\\
  \sigma_{y}(-3) &= \pm0.8\\
  y(0) &= \SI{1.0}{}  \\
  \sigma_{y}(0) &= \pm0.2\\
  y(3) &= \SI{4.0}{} \\
  \sigma_{y}(3) &= \pm0.5
\end{align*}
