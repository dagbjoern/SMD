\newpage
\section{Aufgabe4}
\label{sec:a4}

\subsection{a)}
\label{subsec:a4a}
Gegeben: 2D-Gaussverteilung mit den Parametern:
\\
\begin{align*}
\mu_{x} &= 4\\
\mu_{y} &= 2\\
\sigma_{x} &= 3.5\\
\sigma_{y} &= 1.5\\
Cov\left( x,y \right) &= 4.2
\end{align*}

Der Korrelationskoeffizient $\rho\left( x,y \right)$ ist gegeben durch:
\\
\begin{equation}
  \label{eqn:korkoef}
  \rho\left( x,y \right) = \frac{Cov\left( x,y \right)}{\sigma_{x} \cdot \sigma_{y}}.
\end{equation}

Mit den Werten für $\sigma_{x}$  und $\sigma_{y}$ ergibt sich über Gleichung \eqref{eqn:korkoef}
\\
\begin{align*}
    \rho\left( x,y \right) = 0.8
\end{align*}


\subsection{b)}
\label{subsec:a4b}
Die 2D-Gaussverteilung hat die Dichtefunktion:
\\
\begin{equation}
  \label{eqn:gaussdichte}
  \Phi =  k \cdot e^{-0.5 \left( \cdot \left( \symbf{X} - \symbf{\mu} \right)^{\top} \symbf{B} \left( \symbf{X} - \symbf{\mu} \right)\right)}
\end{equation}
Dabei ist die Matrix $B$ gegeben durch :
\\
\begin{equation}
  \label{eqn:B}
  \symbf{B} = \frac{1}{\sigma_{x}^{2} \cdot \sigma_{y}^{2} - Cov\left( x,y \right)^{2}} \cdot
  \begin{pmatrix}
    \sigma_{y}^{2} & -Cov\left( x,y \right)\\
    -Cov\left( x,y \right) & \sigma_{x}^{2}
  \end{pmatrix}
\end{equation}
Durch Einsetzen von \eqref{eqn:B} in Beziehung \eqref{eqn:gaussdichte} sowie einige Umformungsschritte und anschließendes Logarithmieren ergibt sich für die Dichtefunktion(soll als konstant betrachtetn werden):
\\
\begin{equation}
  \label{gaussdichte2}
  \ln\left(\frac{\Phi}{k} \right) = \frac{1}{1-\rho^{2}} \cdot \left( \frac{\left( x - \mu_{x} \right)^{2}}{\sigma_{x}^{2}} - 2 \cdot
  \rho \frac{\left( x - \mu_{x} \right) \cdot \left( y - \mu_{y} \right)}{\sigma_{x} \cdot \sigma_{y}} +
  \frac{\left( y - \mu_{y} \right)^{2}}{\sigma_{y}^{2}} \right)
\end{equation}

Diese Gleichung stellt eine in $x$ und $y$ quadratische Funktion dar, die gleich einer Konstanten ist. Es handelt sich um eine Ellipsengleichung.
\\

\subsection{c)}
\label{subsec:a4c}

\subsection{d)}
\label{subsec:a4d}
\subsection{e)}
\label{subsec:a4e}
Der Winkel den die Hauptachsen der Kovarianzellipse mit den Hauptachsen
einschließen, ist gegeben durch:
\\
\begin{equation}
  \label{eqn:winkel}
  \alpha = 0.5 \cdot \arctan\left( \frac{2\rho\sigma_{x}\sigma_{y}}{\sigma_{x}^{2} - \sigma{y}^{2}} \right) = \SI{0.349}{\radian}
\end{equation}
Mit dem Winkel $\alpha$ lassen sich nun die Längen der Hauptachsen bestimmen:
\\
\begin{align}
  \H_\text{l} &= \sqrt{\left( 1 - \rho^{2} \right) \cdot \frac{1}{\frac{\cos\left( \alpha \right)^{2} - \frac{2\rho\sin\left( \alpha \right) \cos\left( \alpha \right)}{\sigma_{x} \cdot \sigma_{y}} +
  \frac{\sin\left( \alpha \right)^{2}}{\sigma_{y}^{2}}}{\sigma_{x}^{2}}}}
  &= 27.560\\
  \H_\text{k} &= \sqrt{\left( 1 - \rho^{2} \right) \cdot \frac{1}{\frac{\sin\left( \alpha \right)^{2} - \frac{2\rho\sin\left( \alpha \right) \cos\left( \alpha \right)}{\sigma_{x} \cdot \sigma_{y}} +
  \frac{\cos\left( \alpha \right)^{2}}{\sigma_{y}^{2}}}{\sigma_{x}^{2}}}}
  &= 1.184
\end{align}

\subsection{f)}
\label{subsec:a4f}
Die bedingten Wahrscheinlichkeitsdichten, lassen sich über
\\
\begin{align}
  \label{eqn:bedwahr}
  \Phi(x\mid y_\text{c}) &= \frac{\Phi \left( x,y \right)}{h \left( y \right)}\\
  \Phi(y\mid x_\text{c}) &= \frac{\Phi \left( x,y \right)}{g \left( x \right)}
\end{align}
berechnen. Dabei sind $g$ , $h$ die Dichten der Randverteilungen und $y_\text{c}$ bzw.$x_\text{c}$ sollen anzeigen, dass es jeweils für diese Variable ein Wert
eingesetzt wird. Es gilt:
\\
\begin{align}
  g\left( x \right) &= \int_{-\infty}^{\infty} \Phi\left( x,y \right) \mathrm{d}y\\
  h\left( y \right) &= \int_{-\infty}^{\infty} \Phi\left( x,y \right) \mathrm{d}x
\end{align}
Es ergeben sich für die bedingten Wahrscheinlichkeiten
\\
\begin{align}
  \Phi\left( x\mid y_\text{c} \right) &= \frac{1}{\sqrt{2\pi \cdot \sigma_{x} \cdot\left( 1 - \rho^{2} \right)}}
  \cdot e^{\frac{-1}{2\cdot\left( 1-\rho^{2} \right)} \cdot \left( \frac{x - \mu_{x}}{\sigma_{x}} - \frac{\left( y_\text{c} -  \right)}{\sigma_{y}} \right)}\\
  \Phi\left( y\mid x_\text{c} \right) &= \frac{1}{\sqrt{2\pi \cdot \sigma_{y} \cdot\left( 1 - \rho^{2} \right)}}
  \cdot e^{\frac{-1}{2\cdot\left( 1-\rho^{2} \right)} \cdot \left( \frac{y - \mu_{y}}{\sigma_{y}} - \frac{\left( x_\text{c} -  \right)}{\sigma_{x}} \right)}
\end{align}

\subsection{g)}
\label{subsec:a4g}
Für die bedingten Erwartungswerte in $x$ und $y$ gilt
\\
\begin{align}
  E\left( x \mid y_\text{c} \right) &= \int_{-\infty}^{\infty} x \cdot
  \left( x \mid y_\text{c} \right) \mathrm{d}x \\
  E\left( x \mid y_\text{c} \right) &= \int_{-\infty}^{\infty} x \cdot
  \left( x \mid y_\text{c} \right) \mathrm{d}x
\end{align}
Durch Integration ergeben sich:
\\
\begin{align}
  E\left( x \mid y_\text{c} \right) &= \mu_{x} + \sigma_{x} \cdot
  \rho \cdot \left( \frac{y_\text{c} - \mu_{y}}{\sigma_{y}} \right) &= 4 + \frac{2.8 \cdot\left( y_\text{c} - 2 \right)}{1.5}\\
  E\left( y \mid x_\text{c} \right) &= \mu_{y} + \sigma_{y} \cdot
  \rho \cdot \left( \frac{x_\text{c} - \mu_{x}}{\sigma_{x}} \right) &= 2 + \frac{1.2 \cdot\left( x_\text{c} - 4\right)}{3.5}
\end{align}
