\newpage
\section{Aufgabe2}
\label{sec:a2}

\subsection{a)}
\label{subsec:a2a}

\begin{figure}
  \centering
  \includegraphics[width=\textwidth]{plotge2a).pdf}
  \caption{Histogramme für die Gewichtsverteilungen mit Unterschiedlicher Bins Anzahl.}
  \label{fig:gr}
\end{figure}
\FloatBarrier


\begin{figure}
  \centering
  \includegraphics[width=\textwidth]{plotgr2a).pdf}
  \caption{Histogramme für die Größenverteilungen mit Unterschiedlicher Bins Anzahl.}
  \label{fig:ge}
\end{figure}
\FloatBarrier
Bei einer zu hohen Bin-Anzahl entstehen leere Bins zwischen

\subsection{b)}
\label{subsec:a2b}
Bei Daten von mehr als 250 Personen kann es bei zu Breit gewählten Bins sein,
dass die Bins in der Mitte eine deutlich größere Anzahl enthalten als die Außeren.
Was dazu führen könnte, dass
die Bins außerhalb deutlich kleiner sind als die in der Mitte der Normalverteilung.
Um dies zu verhinder muss die Bin bereite verkleinert werden. Jedoch darf
diese nicht zu klein gewählt werden, da sonst wieder leere Bins entstehen
können.
Sinnvolle minimale Bin-Breiten :
\begin{align*}
  \intertext{Größe:}
    1
  \intertext{Gewicht:}
   0,01
\end{align*}
Position der Bin-Mitte genau in der Mitter der Breite
\subsection{c)}
\label{subsec:a2c}

\begin{figure}
  \centering
  \includegraphics[width=\textwidth]{plotgr2a).pdf}
  \caption{Histogramme für $10^5$ logarithmierte gleichverteilte Zahlen aus
  dem Intervall $1-100$ mit Unterschiedlicher Bins Anzahl.}
  \label{fig:ge}
\end{figure}
\FloatBarrier

Bei zu hoher Bin-Zahl entstehten Leere Bins in dem Bereich zwischen 0 und 2.
Folglich
