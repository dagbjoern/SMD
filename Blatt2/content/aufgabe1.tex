\newpage
\section{Aufgabe 1}
\label{sec:a1}

\subsection{a)}
\label{subsec:a1a}
Wenn die Zufallsvariable $x$ mit der Wahrscheinlichkeitsdichte
\\
\begin{equation}
  \label{eqn:dichte}
  f\left( x \right) =
  \begin{cases}
     1 , & 0 \le x \le 1 \\
     0 , & \text{sonst}
  \end{cases}
\end{equation}
beschrieben werden kann, lässt sich die Wahrscheinlichkeit, $x$ im Intervall von $a$
bis $b$ zu erhalten, durch
\\
\begin{equation}
  \label{eqn:wkeit}
  p\left( a \le x \le b \right) = \int_{a}^{b} 1 \mathrm{d}x
\end{equation}
bestimmen. Dies gilt, obwohl sich die Integrationsgrenzen eigentlich vom negativen Unendlichem
bis zum positiven Undendlichen erstrecken, weil außerhalb des Intervalls ist $f\left(x\right)$ gleich $0$.


Somit beträgt die Wahrscheinlichkeit für das Intervall zwischen $\frac{1}{3}$ und $\frac{1}{2}$
\\
\begin{align*}
  p\left( \frac{1}{3} \le x \le \frac{1}{2} \right) &= \int_{\frac{1}{3}}^{\frac{1}{3}}
  1 \mathrm{d}x \\
  &= \frac{1}{6}.
\end{align*}

\subsection{b)}
\label{subsec:a1b}
Die Warscheinlichkeit $x$ bei genau $\frac{1}{2}$ zu finden ist gleich $0$ , denn wenn
man die Zahl nicht in einem Intervall sondern an einer bestimmten Stelle
sucht ergibt sich gemäß Gleichung \eqref{eqn:wkeit} der Wert $0$. Dieses Ergebnis lässt sich auch dadurch
erklären, dass es zwischen $0$ und $1$ unendlich viele Zahlen gibt. Somit ergibt sich bei Gleichverteilung
als Wahrscheinlichkeit für eine bestimmte Zahl der Wert $0$.

\subsection{c)}
\label{subsec:a1c}
Es wird eine Gleitkommazahl mit einer $23$-stelligen Mantisse betrachtet.
Die größte allein mit der Mantisse darstellbare Dezimalzahl ist
\\
\begin{align*}
  2^{23} - 1 = 8388607.
\end{align*}
Da der Computer Zahlen somit also Dezimalzahlen die in der ersten Nachkommastelle eine $5$, in den sechs
Folgenden Stellen eine $0$ und daraus folgend Nachkommastellen ungleich $0$ besitzen fälschlicherweise als
$0.5$ interpretiert gibt es also mehrere mögliche Generator-Ergebnisse die auf den für den Computer exakten Wert
$0.5$ führen. Der Fehler der dabei gemacht wird befindet sich also in der $8.$ Nachkommastelle. Dies führt auf
einen maximalen Abstand eines fälschlicherweise als $0.5$ interpretierten Werts von der Zahl $0.5$ von
\\
\begin{equation}
  \label{eqn:abstand}
  \varepsilon = 10^{-7}.
\end{equation}

 Somit ergibt sich gemäß Gleichung \eqref{eqn:wkeit}
 \\
 \begin{align*}
   \tilde{p}\left( x = \frac{1}{2}\right) &= \int_{\frac{1}{2}}^{\frac{1}{2} + \varepsilon} 1 \mathrm{d}x\\
   &= 10^{-7}
 \end{align*}


\subsection{d)}
\label{subsec:a1d}
Die zahl $\frac{2}{3}$ kann bei endlichem Speicherplatz nicht exakt als Gleitkommazahl dargestellt werden.
Somit ist die Wahrscheinlichkeit die Zahl $2/3$ mit dem in dieser Aufgabe beschriebenen Generator exakt zu erhalten
identisch mit $0$.
