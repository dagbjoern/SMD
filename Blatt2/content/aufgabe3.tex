\newpage
\section{Aufgabe3}
\label{sec:a3}
Der Zufallszahlengenerator Z liefert gleichverteilte Zahlen von $0$ bis $1$
Um nun Zufallszahlen zu erzeugen,die den folgenden Verteilungen $f(x)$ gehorchen,
mussen die Werte von Z in die Funktion $x(r)$ für r eingesetzt werden.


\subsection{a)}
\label{subsec:a3a}
$f(x)=N$ in der Grenzen $x_\text{min}$ bis $s_\text{max}$.

\begin{align}
A&=\int_{x_\text{min}}^{x_\text{max}} N dx=
N(x_\text{max}-x_\text{min})\stackrel{!}{=}1\\
\Rightarrow N&=\frac{1}{(x_\text{max}-x_\text{min})}\\
A(x)&=\int_{x_\text{min}}^{x}\frac{1}{(x_\text{max}-x_\text{min})}dx=\frac{x_\text{min}}{x_\text{max}-x_\text{min}}\\
r(x)&=\frac{A(x)}{A}=\frac{x_\text{min}}{x_\text{max}-x_\text{min}}\\
\iff x(r)&=r(x_\text{max}-x_\text{min})+x_\text{min}
\end{align}


\begin{figure}
  \centering
  \includegraphics[width=\textwidth]{Gleichverteilung.pdf}
  \caption{Histogramm für eine Gleichverteilung in den Grenzen $x_\text{min}=10$ bis $x_\text{max}=100$ mit 100 Bins.}
  \label{fig:gleich}
\end{figure}
\FloatBarrier

\subsection{b)}
\label{subsec:a3b}
$f(t)=N\symup{e}^{-\frac{t}{\tau}}$ in den Grenzen $0$ bis $\infty$.

\begin{align}
A&=\int_0^{\infty}N\symup{e}^{-\frac{t}{\tau}}dt=N\tau\stackrel{!}{=}1\\
\Rightarrow N&=\frac{1}{\tau}\\
A(x)&=\int_0^x \frac{1}{\tau}\symup{e}^{-\frac{t}{\tau}}dt=-\symup{e}^{-\frac{x}{\tau}+1}\\
r(x)&=\frac{A(x)}{A}=-\symup{e}^{-\frac{x}{\tau}+1}\\
\iff x(r)&=-\tau\ln(1-r)
\end{align}


\begin{figure}
  \centering
  \includegraphics[width=\textwidth]{Exponentialgesetz.pdf}
  \caption{Histogramm für ein Verteilung die dem Exponentialgesetz mit $\tau=10$ unterliegt mit 100 Bins.}
  \label{fig:expo}
\end{figure}
\FloatBarrier

\subsection{c)}
\label{subsec:a3c}
$f(x)=Nx^{-n}$ in den Grenzen  $x_\text{min}$ bis $s_\text{max}$.

\begin{align}
A&=\int_{x_\text{min}}^{x_\text{max}} Nx^{-n} dx=
N(\frac{1}{1-n}x_\text{max}^{1-n}-\frac{1}{1-n}x_\text{min}^{1-n})\stackrel{!}{=}1\\
\Rightarrow N&=\frac{1-n}{x_\text{max}^{1-n}-x_\text{min}^{1-n}}\\
A(x)&=\int_{x_\text{min}}^x \frac{1-n}{x_\text{max}^{1-n}-x_\text{min}^{1-n}}x^-n dx=\frac{1}{x_\text{max}^{1-n}-x_\text{min}^{1-n}}\left(x^{1-n}-x_\text{min}^1-n\right)\\
r(x)&=\frac{A(x)}{A}=\frac{1}{x_\text{max}^{1-n}-x_\text{min}^{1-n}}\left(x^{1-n}-x_\text{min}^1-n\right)\\
\iff x(r) &=\sqrt[1-n]{r(x_\text{max}^{1-n}-x_\text{min}^{1-n})+x_\text{min}^{1-n}}
\end{align}


\begin{figure}
  \centering
  \includegraphics[width=\textwidth]{Potenzgesetz.pdf}
  \caption{Histogramm für ein Verteilung die dem Potenzgesetz mit $n=9$ unterliegt in den Grenzen $x_\text{min}=10$ bis $x_\text{max}=100$  mit 100 Bins.}
  \label{fig:poten}
\end{figure}
\FloatBarrier



\subsection{d)}
\label{subsec:a3d}
$f(x)=\frac{1}{\pi}\frac{1}{1+x^{2}} $ in den Grenzen $-\infty$ bis $\infty$
\begin{align}
A&=\int_{-\infty}^{\infty} \frac{1}{\pi}\frac{1}{1+x^{2}}dx =\frac{1}{\pi} \arctan(x)\bigr|_{-\infty}^{\infty}=1 \\
A(x)&=\int_{-\infty}^{x} \frac{1}{\pi}\frac{1}{1+x^{2}}dx =\frac{1}{\pi} \arctan(x)\bigr|_{-\infty}^{x}
=\frac{\arctan(x)}{\pi}+\frac{1}{2}\\
r(x)&=\frac{A(x)}{A}=\frac{\arctan(x)}{\pi}+\frac{1}{2}\\
\iff x(r)&=\tan\left(\left(r-\frac{1}{2}\right)\pi\right)
\end{align}

\begin{figure}
  \centering
  \includegraphics[width=\textwidth]{CauchyVerteilung.pdf}
  \caption{Histogramm für eine Cauchy-Verteilung mit 10000 Bins.}
  \label{fig:cauchy}
\end{figure}

\subsection{e)}
\label{subsec:a3e}
