\newpage
\section{Aufgabe4}
\label{sec:a4}

Die Normierungskonstante $N$ ist bei $f(v)$ gleich 1
\subsection{a)}
\label{subsec:a1a}
Für wahrscheinlichste Geschwindigkeit $v_m$ muss
$f'(v)$ bestimmt werden:
\begin{align}
  f'(v)=\left(\frac{m}{2\pi k_B T}\right)^{3/2}\left(\frac{4\pi v}{k_BT} e^{-\frac{mv^2}{2 k_B T}}\left(-mv^2+2k_BT\right) \right)
\intertext{und aus}
f'(v)=0
\intertext{folgt}
v_m=\sqrt{\frac{2kT}{m}}
\end{align}

\subsection{b)}
\label{subsec:a1b}
Für den Erwartungswert $\Braket{v}$ muss das Intergral:
\begin{align}
  \Braket{v}&=\int_0^{\infty} f(v) v \symup{d} v\\
&=\left(\frac{m}{2\pi k_B T}\right)^{3/2}4\pi \int_0^{\infty} \underbrace{v e^{-\frac{mv^2}{2 k_B T}}}_{u'}
\underbrace{v^2}_{g} \symup{d}v\\
u=-\frac{k_BT}{m}e^{-\frac{mv^2}{2 k_B T}}   & \: \: v'=2v
\intertext{Mit Hilfe dieses Ansatzes für die partielle Intergration ergibt sich:}
\Braket{v}&=\sqrt{\frac{8 k_B T}{m \pi}}
\end{align}

\subsection{c)}
Um der Median der Geschwindigkeit $v_{0,5}$ in Abhängigkeit von $v_m$ zu berechne, muss zu nächst $f(v)$
mit
\begin{align}
  \frac{k_BT}{m}=2v_m
\end{align}
umgeformt werden.
\begin{align}
  \Rightarrow f(v)=\left(\frac{1}{4\pi v_m^2}\right)^{\frac{2}{3}}e^{-\frac{v^2}{4v_m^2}}4\pi v^2
\end{align}
\subsection{d)}


\subsection{e)}
Für die Standardabweichung der Geschwindigkeit $\sigma_v$ muss zunächst
$\Braket{v^2}$ berechnet werden.
\begin{align}
\Braket{v^2}&=\int_0^{\infty} f(v) v^2 \symup{d} v\\
\intertext{Das Integral
lässt sich mit mehrfacher partieller
Intergration in der Form, die in \ref{subsec:a1b} verwendet wird, lösen und es ergibt sich:}
\Braket{x^2}&=6\sqrt{\frac{2}{\pi}}\left(\frac{k_BT}{m}\right)^{\frac{3}{2}}
\intertext{Für die Standardabweichung folgt:}
\sigma_v&=\sqrt{\Braket{v^2}-\Braket{v}^2} =\sqrt{6\sqrt{\frac{2}{\pi}}\left(\frac{k_BT}{m}\right)^{\frac{3}{2}} -\frac{8k_BT}{m\pi}}
\end{align}
